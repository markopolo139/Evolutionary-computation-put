\begin{table}[!htbp]
\centering
\caption{Running times in milliseconds}
\begin{tabular}{lcc}
\hline
 Method & TSPA & TSPB \\
\hline
ls\_steepest\_edges\_random & 16.1544 (13.3485 - 32.097) & 16.2354 (12.678 - 26.6363) \\
ls\_improving\_moves\_list\_random & 8.81557 (5.42845 - 15.1835) & 9.06173 (5.74288 - 13.987) \\
\hline
\end{tabular}
\end{table}