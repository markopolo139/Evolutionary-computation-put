\documentclass{article}

\usepackage[english]{babel}
\usepackage[letterpaper,top=2cm,bottom=2cm,left=3cm,right=3cm,marginparwidth=1.75cm]{geometry}

\usepackage{graphicx}
\usepackage[colorlinks=true, allcolors=blue]{hyperref}
\usepackage{tabularx}
\usepackage{booktabs}
\usepackage{multirow}
\usepackage{ragged2e}

\usepackage{algorithm}
\usepackage{algpseudocode}
\usepackage{amsmath}

\title{Evolutionary Computation - Assignment 3}
\author{
  Patryk Janiak\\
  \texttt{156053}
  \and
  Marek Seget\\
  \texttt{156042}
}

\begin{document}
\maketitle

\section{Problem description}
We are given three columns of integers with a row for each node. The first two columns contain x
and y coordinates of the node positions in a plane. The third column contains node costs. The goal is
to select exactly 50\% of the nodes (if the number of nodes is odd we round the number of nodes to
be selected up) and form a Hamiltonian cycle (closed path) through this set of nodes such that the
sum of the total length of the path plus the total cost of the selected nodes is minimized.

\section{Local Search Algorithms}

The following section describes the local search algorithms used to solve the TSP instances. The 8 different methods implemented are combinations of two main local search strategies (Steepest and Greedy), two different neighborhood definitions (Node-based and Edge-based), and two different starting solution generation methods (Random and Greedy Construction).

\subsection{Local Search Strategies}

Two main local search strategies were implemented: Steepest Local Search and Greedy Local Search.

\begin{algorithm}
\caption{Steepest Local Search}
\label{alg:steepest_ls}
\begin{algorithmic}[1]
\Require Initial solution $S$, Neighborhood structure $N(S)$
\Ensure Optimized solution $S$
\State $improvement \gets \text{true}$
\While{$improvement$}
    \State $improvement \gets \text{false}$
    \State Find move $m^* \in N(S)$ that maximizes $-\Delta(m)$
    \If{$\Delta(m^*) < 0$}
        \State $S \gets \text{apply\_move}(S, m^*)$
        \State $improvement \gets \text{true}$
    \EndIf
\EndWhile
\State \Return $S$
\end{algorithmic}
\end{algorithm}

\begin{algorithm}
\caption{Greedy Local Search}
\label{alg:greedy_ls}
\begin{algorithmic}[1]
\Require Initial solution $S$, Neighborhood structure $N(S)$
\Ensure Optimized solution $S$
\State $improvement \gets \text{true}$
\While{$improvement$}
    \State $improvement \gets \text{false}$
    \State Randomly shuffle the order of moves in $N(S)$
    \For{each move $m$ in the shuffled neighborhood}
        \If{$\Delta(m) < 0$}
            \State $S \gets \text{apply\_move}(S, m)$
            \State $improvement \gets \text{true}$
            \State \textbf{break}
        \EndIf
    \EndFor
\EndWhile
\State \Return $S$
\end{algorithmic}
\end{algorithm}

\newpage

\subsection{Neighborhood Structures}

The neighborhood of a solution is defined by the set of possible moves that can be applied to it. Two types of neighborhood structures were used:

\begin{itemize}
    \item \textbf{Node-based Neighborhood:}
    \begin{itemize}
        \item \textbf{Intra-route moves:} Swapping any two nodes within the solution.
        \item \textbf{Inter-route moves:} Swapping any node in the solution with any node not in the solution.
    \end{itemize}
    \item \textbf{Edge-based Neighborhood:}
    \begin{itemize}
        \item \textbf{Intra-route moves:} 2-opt moves, which consist of swapping two edges in the tour.
        \item \textbf{Inter-route moves:} Swapping any node in the solution with any node not in the solution.
    \end{itemize}
\end{itemize}

\subsection{Starting Solutions}

Two methods were used to generate the initial solutions for the local search algorithms:

\begin{itemize}
    \item \textbf{Random Start:} A random solution is generated by shuffling all available points and selecting the first half.
    \item \textbf{Greedy Start:} A solution is constructed using the \textit{Nearest Neighbor with Weighted Regret} heuristic.
\end{itemize}

\subsection{Implemented Methods}

The 8 implemented methods are combinations of the above strategies and structures:

\begin{itemize}
    \item ls\_steepest\_nodes\_random: Steepest search with a node-based neighborhood and a random starting solution.
    \item ls\_steepest\_nodes\_greedy: Steepest search with a node-based neighborhood and a greedy starting solution.
    \item ls\_steepest\_edges\_random: Steepest search with an edge-based neighborhood and a random starting solution.
    \item ls\_steepest\_edges\_greedy: Steepest search with an edge-based neighborhood and a greedy starting solution.
    \item ls\_greedy\_nodes\_random: Greedy search with a node-based neighborhood and a random starting solution.
    \item ls\_greedy\_nodes\_greedy: Greedy search with a node-based neighborhood and a greedy starting solution.
    \item ls\_greedy\_edges\_random: Greedy search with an edge-based neighborhood and a random starting solution.
    \item ls\_greedy\_edges\_greedy: Greedy search with an edge-based neighborhood and a greedy starting solution.
\end{itemize}

\section{Results}

\begin{table}[!htbp]
\centering
\caption{Objective function scores}
\begin{tabular}{lcc}
\hline
 Method & TSPA & TSPB \\
\hline
ls\_steepest\_edges\_random & 74054.4 (71611 - 77589) & 48234.7 (46265 - 51786) \\
ls\_candidate\_random & 79827.4 (74247 - 88032) & 49034.9 (46938 - 53260) \\
\hline
\end{tabular}
\end{table}

\begin{table}[!htbp]
\centering
\caption{Running times in milliseconds}
\begin{tabular}{lcc}
\hline
 Method & TSPA & TSPB \\
\hline
ls\_steepest\_edges\_random & 15.9243 (13.0127 - 20.4555) & 15.5694 (13.0003 - 20.0678) \\
ls\_candidate\_random & 11.2681 (9.50151 - 13.7007) & 12.651 (10.4833 - 19.6281) \\
\hline
\end{tabular}
\end{table}

\subsection{TSPA.csv}

\begin{table}[H] % Use [H] from float package for 'here' if available, otherwise use [!htbp]
\centering
\caption{Best Achieved Solution for TSPA.csv}
\label{tab:tsbp_best_sequence}
\begin{tabular}{|p{0.25\linewidth}|p{0.75\linewidth}|} % Two columns: 25% for Method, 75% for Sequence
\hline
\textbf{Method} & \textbf{Best Sequence} \\
\hline
ls\_steepest\_edges\_random & 49, 102, 14, 144, 62, 9, 15, 186, 89, 183, 143, 0, 117, 68, 46, 115, 139, 41, 193, 159, 108, 18, 22, 146, 34, 48, 54, 177, 10, 184, 160, 181, 42, 5, 43, 116, 65, 149, 59, 118, 51, 151, 162, 123, 127, 70, 135, 154, 180, 53, 63, 79, 133, 80, 176, 137, 148, 124, 94, 152, 2, 1, 97, 26, 100, 86, 101, 75, 120, 44, 82, 129, 92, 179, 145, 78, 25, 16, 171, 175, 113, 56, 31, 157, 196, 81, 90, 27, 164, 95, 39, 165, 119, 40, 185, 57, 55, 52, 106, 178, 49 \\
\hline
ls\_candidate\_random & 40, 165, 106, 178, 52, 55, 57, 92, 129, 2, 152, 1, 75, 86, 101, 97, 26, 100, 121, 53, 180, 158, 154, 70, 135, 162, 123, 127, 4, 112, 84, 184, 177, 10, 54, 160, 34, 146, 22, 18, 108, 159, 193, 41, 96, 5, 42, 43, 116, 65, 47, 131, 59, 51, 109, 118, 115, 198, 139, 68, 46, 0, 143, 183, 89, 186, 23, 137, 176, 80, 151, 133, 79, 63, 94, 148, 62, 9, 144, 155, 49, 14, 138, 21, 164, 90, 81, 196, 31, 113, 56, 175, 171, 16, 44, 120, 25, 78, 145, 185, 40 \\
\hline
\end{tabular}
\end{table}

\begin{figure}[!htbp]
\centering
\includegraphics[width=0.97\linewidth]{TSPA_solution_ls_steepest_edges_random.png}
\caption{\label{fig:frog}ls\_steepest\_edges\_random}
\end{figure}

\begin{figure}[!htbp]
\centering
\includegraphics[width=0.97\linewidth]{TSPA_solution_ls_candidate_random.png}
\caption{\label{fig:frog}ls\_candidate\_random}
\end{figure}

\subsection{TSPB.csv}

\begin{table}[H] % Use [H] from float package for \'here\' if available, otherwise use [!htbp]
\centering
\caption{Best Achieved Solution for TSPB.csv}
\label{tab:tsbp_best_sequence}
\begin{tabular}{|p{0.25\linewidth}|p{0.75\linewidth}|} % Two columns: 25% for Method, 75% for Sequence
\hline
\textbf{Method} & \textbf{Best Sequence} \\
\hline
ls\_steepest\_nodes\_random & 160, 0, 109, 83, 18, 62, 124, 141, 91, 11, 168, 195, 169, 90, 156, 198, 54, 13, 145, 70, 161, 3, 33, 144, 177, 5, 175, 78, 80, 190, 117, 1, 27, 38, 131, 138, 128, 185, 66, 94, 179, 153, 81, 143, 34, 55, 106, 56, 104, 8, 58, 77, 187, 163, 103, 113, 166, 99, 140, 152, 155, 15, 6, 147, 10, 133, 125, 51, 121, 112, 193, 164, 73, 31, 136, 45, 36, 61, 21, 82, 111, 183, 130, 95, 86, 194, 176, 180, 127, 89, 25, 135, 63, 100, 122, 118, 139, 12, 35, 29, 160 \\
\hline
ls\_steepest\_nodes\_greedy & 40, 107, 100, 63, 122, 135, 38, 27, 16, 1, 156, 198, 117, 193, 54, 31, 73, 136, 190, 80, 162, 175, 78, 5, 177, 25, 182, 138, 11, 29, 109, 35, 0, 160, 33, 104, 8, 82, 21, 36, 61, 91, 141, 77, 81, 153, 187, 163, 89, 127, 103, 113, 176, 194, 166, 86, 185, 179, 172, 57, 66, 94, 47, 148, 60, 20, 28, 149, 4, 140, 183, 130, 95, 128, 106, 124, 62, 18, 55, 34, 170, 152, 184, 155, 3, 70, 15, 145, 195, 168, 13, 132, 169, 188, 6, 147, 90, 51, 121, 131, 40 \\
\hline
ls\_steepest\_edges\_random & 95, 185, 86, 166, 194, 176, 180, 113, 114, 137, 127, 89, 103, 163, 153, 77, 97, 141, 91, 36, 61, 21, 177, 5, 78, 175, 45, 80, 190, 73, 54, 31, 193, 117, 198, 156, 27, 38, 16, 1, 131, 112, 121, 51, 90, 122, 135, 63, 100, 40, 107, 72, 133, 10, 147, 85, 134, 6, 188, 169, 132, 70, 3, 155, 189, 15, 145, 195, 168, 139, 11, 138, 33, 160, 144, 56, 104, 8, 111, 29, 0, 109, 35, 106, 124, 62, 18, 55, 183, 140, 28, 20, 60, 148, 47, 94, 179, 22, 99, 130, 95 \\
\hline
ls\_steepest\_edges\_greedy & 147, 6, 188, 169, 132, 13, 195, 168, 145, 15, 70, 3, 155, 184, 152, 170, 34, 55, 18, 62, 124, 106, 128, 95, 130, 183, 140, 4, 149, 28, 20, 60, 148, 47, 94, 66, 57, 172, 179, 185, 86, 166, 194, 176, 113, 103, 127, 89, 163, 129, 153, 81, 77, 141, 91, 36, 61, 21, 82, 8, 104, 111, 35, 109, 0, 29, 160, 33, 11, 139, 138, 182, 25, 177, 5, 78, 175, 45, 80, 190, 136, 73, 54, 31, 193, 117, 198, 1, 131, 121, 51, 90, 122, 135, 63, 100, 40, 107, 133, 10, 147 \\
\hline
ls\_greedy\_nodes\_random & 99, 86, 113, 26, 127, 163, 187, 146, 141, 91, 79, 61, 162, 46, 108, 42, 27, 16, 131, 121, 158, 192, 150, 6, 43, 138, 104, 56, 8, 58, 87, 142, 136, 54, 173, 123, 177, 77, 81, 106, 128, 159, 41, 111, 0, 109, 69, 184, 15, 145, 169, 188, 147, 67, 120, 51, 98, 134, 2, 139, 49, 29, 62, 83, 55, 154, 47, 101, 4, 152, 71, 191, 178, 122, 44, 133, 115, 10, 17, 72, 100, 63, 96, 32, 135, 24, 198, 193, 31, 171, 157, 160, 34, 140, 23, 60, 57, 172, 179, 95, 99 \\
\hline
ls\_greedy\_nodes\_greedy & 131, 121, 51, 90, 147, 6, 188, 169, 132, 13, 195, 168, 145, 15, 70, 3, 155, 184, 152, 170, 34, 55, 18, 62, 124, 106, 128, 95, 130, 183, 140, 4, 149, 28, 20, 60, 148, 47, 94, 66, 57, 172, 179, 185, 86, 166, 194, 176, 113, 103, 127, 89, 163, 129, 153, 81, 77, 141, 91, 36, 61, 21, 82, 8, 104, 33, 160, 0, 35, 109, 29, 11, 138, 182, 25, 177, 5, 78, 175, 162, 80, 190, 136, 73, 54, 31, 193, 117, 198, 1, 16, 27, 38, 63, 40, 107, 100, 122, 135, 131 \\
\hline
ls\_greedy\_edges\_random & 45, 80, 190, 136, 73, 54, 31, 193, 117, 198, 1, 38, 63, 40, 107, 100, 135, 122, 133, 10, 90, 131, 121, 51, 71, 147, 134, 6, 188, 169, 132, 13, 195, 168, 145, 15, 70, 3, 155, 189, 69, 109, 0, 35, 143, 106, 124, 62, 18, 55, 34, 170, 152, 183, 140, 4, 149, 28, 20, 60, 148, 47, 94, 179, 185, 86, 166, 194, 176, 180, 113, 114, 137, 127, 165, 89, 103, 26, 163, 153, 77, 111, 144, 29, 160, 33, 11, 138, 104, 8, 82, 21, 141, 61, 36, 177, 5, 78, 175, 142, 45 \\
\hline
ls\_greedy\_edges\_greedy & 147, 6, 188, 169, 132, 13, 195, 168, 145, 15, 70, 3, 155, 184, 152, 170, 34, 55, 18, 62, 124, 106, 128, 95, 130, 183, 140, 4, 149, 28, 20, 60, 148, 47, 94, 66, 57, 172, 179, 185, 86, 166, 194, 176, 113, 103, 127, 89, 163, 187, 153, 81, 77, 141, 91, 36, 61, 21, 82, 8, 104, 111, 35, 109, 0, 29, 160, 33, 11, 139, 138, 182, 25, 177, 5, 78, 175, 45, 80, 190, 136, 73, 54, 31, 193, 117, 198, 1, 131, 121, 51, 90, 122, 135, 63, 100, 40, 107, 133, 10, 147 \\
\hline
\end{tabular}
\end{table}


\begin{figure}[!htbp]
\centering
\includegraphics[width=0.97\linewidth]{TSPB_solution_ls_steepest_nodes_random.png}
\caption{\label{fig:frog}ls\_steepest\_nodes\_random}
\end{figure}

\begin{figure}[!htbp]
\centering
\includegraphics[width=0.97\linewidth]{TSPB_solution_ls_steepest_nodes_greedy.png}
\caption{\label{fig:frog}ls\_steepest\_nodes\_greedy}
\end{figure}

\begin{figure}[!htbp]
\centering
\includegraphics[width=0.97\linewidth]{TSPB_solution_ls_steepest_edges_random.png}
\caption{\label{fig:frog}ls\_steepest\_edges\_random}
\end{figure}

\newpage

\begin{figure}[!htbp]
\centering
\includegraphics[width=0.97\linewidth]{TSPB_solution_ls_steepest_edges_greedy.png}
\caption{\label{fig:frog}ls\_steepest\_edges\_greedy}
\end{figure}

\begin{figure}[!htbp]
\centering
\includegraphics[width=0.97\linewidth]{TSPB_solution_ls_greedy_nodes_random.png}
\caption{\label{fig:frog}ls\_greedy\_nodes\_random}
\end{figure}

\begin{figure}[!htbp]
\centering
\includegraphics[width=0.97\linewidth]{TSPB_solution_ls_greedy_nodes_greedy.png}
\caption{\label{fig:frog}ls\_greedy\_nodes\_greedy}
\end{figure}

\begin{figure}[!htbp]
\centering
\includegraphics[width=0.97\linewidth]{TSPB_solution_ls_greedy_edges_random.png}
\caption{\label{fig:frog}ls\_greedy\_edges\_random}
\end{figure}

\begin{figure}[!htbp]
\centering
\includegraphics[width=0.97\linewidth]{TSPB_solution_ls_greedy_edges_greedy.png}
\caption{\label{fig:frog}ls\_greedy\_edges\_greedy}
\end{figure}


\newpage

\section{Conclusions}

The local search algorithms, particularly those employing a greedy initial solution, demonstrate a significant improvement over simple constructive heuristics. The quality of the starting solution is crucial, as evidenced by the consistently better performance of methods with a greedy start compared to a random start.

Among the neighborhood structures, the edge-based neighborhood (2-opt) proves to be more effective and efficient than the node-based neighborhood. This is reflected in both the solution quality and the computation times.

The `ls\_steepest\_edges\_greedy` method stands out as the most effective algorithm, achieving the best scores for both TSPA and TSPB instances while also being one of the fastest. This suggests that a steepest search strategy combined with an edge-based neighborhood and a good initial solution is a powerful combination for this problem.

In general, the results highlight the importance of a well-chosen combination of initial solution generation, neighborhood structure, and search strategy for solving the TSP effectively.

Link to the source code:

\url{https://github.com/markopolo139/Evolutionary-computation-put/tree/main/ass3}

The best solutions were checked with the provided solution checker.

\end{document}
