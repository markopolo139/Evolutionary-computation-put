\documentclass{article}

\usepackage[english]{babel}
\usepackage[letterpaper,top=2cm,bottom=2cm,left=3cm,right=3cm,marginparwidth=1.75cm]{geometry}

\usepackage{graphicx}
\usepackage[colorlinks=true, allcolors=blue]{hyperref}
\usepackage{tabularx}
\usepackage{booktabs}
\usepackage{multirow}
\usepackage{ragged2e}

\usepackage{algorithm}
\usepackage{algpseudocode}
\usepackage{amsmath}

\title{Evolutionary Computation - Assignment 3}
\author{
  Patryk Janiak\\
  \texttt{156053}
  \and
  Marek Seget\\
  \texttt{156042}
}

\begin{document}
\maketitle

\section{Problem description}
We are given three columns of integers with a row for each node. The first two columns contain x
and y coordinates of the node positions in a plane. The third column contains node costs. The goal is
to select exactly 50\% of the nodes (if the number of nodes is odd we round the number of nodes to
be selected up) and form a Hamiltonian cycle (closed path) through this set of nodes such that the
sum of the total length of the path plus the total cost of the selected nodes is minimized.

\section{Algorithms}
\label{sec:pseudocode}

\subsection{Edge Recombination Crossover}
The crossover operator constructs an offspring by prioritizing edges that exist in the parent solutions. It maintains an adjacency map of neighbors from both parents and greedily selects the next node based on the fewest remaining connections, preserving the structural features of the parents.

\begin{algorithm}[H]
    \caption{Edge Recombination Crossover}
    \begin{algorithmic}[1]
        \Require Parent $P_1$, Parent $P_2$
        \Ensure Offspring $Child$
        \State Build NeighborMap $M$ from edges in $P_1$ and $P_2$
        \State $Child \gets \emptyset$, $Current \gets P_1[0]$
        \State Append $Current$ to $Child$, Mark $Current$ as visited
        \While{$|Child| < |P_1|$}
            \State $Candidates \gets M[Current] \setminus Visited$
            \If{$Candidates \neq \emptyset$}
                \State $Next \gets \arg\min_{n \in Candidates} \{ |M[n] \setminus Visited| \}$
                \If{tie in neighbors count}
                    \State Break tie with shortest distance $D[Current][n]$
                \EndIf
            \Else
                \State $Next \gets$ Random unvisited node
            \EndIf
            \State $Current \gets Next$
            \State Append $Current$ to $Child$, Mark $Current$ as visited
        \EndWhile
        \State \Return $Child$
    \end{algorithmic}
\end{algorithm}

\newpage

\subsection{Double Bridge Mutation}
The mutation operator introduces large-scale perturbations to escape local optima. It applies a random non-sequential 4-opt move (Double Bridge), splitting the tour into four segments and reconnecting them in a scrambled order ($S_1-S_2-S_3-S_4 \to S_1-S_4-S_3-S_2$).

\begin{algorithm}[H]
    \caption{Double Bridge Mutation}
    \begin{algorithmic}[1]
        \Require Solution $S$
        \Ensure Mutated Solution $S'$
        \State Select 3 random cut points $c_1 < c_2 < c_3$ in $S$
        \State Define segments: 
        \State 	$Seg_1 \gets S[0 \dots c_1-1]$
        \State 	$Seg_2 \gets S[c_1 \dots c_2-1]$
        \State 	$Seg_3 \gets S[c_2 \dots c_3-1]$
        \State 	$Seg_4 \gets S[c_3 \dots end]$
        \State $S' \gets Seg_1 + Seg_4 + Seg_3 + Seg_2$
        \State \Return $S'$
    \end{algorithmic}
\end{algorithm}

\newpage

\subsection{Optimized Hybrid Evolutionary Algorithm}
Here the biggest difference is using Binary Tournament to pick the first parent for recombination and the island restart after stagnation with best solution preservation. 
\begin{algorithm}[H]
    \caption{Optimized Hybrid Evolutionary Algorithm}
    \begin{algorithmic}[1]
        \Require Time Limit $T_{max}$, Population Size $N=20$, Candidates $K=40$
        \Ensure Best Solution $S_{best}$

        \State \textbf{Initialization:}
        \State $Pop \gets \emptyset$
        \While{$|Pop| < N$}
            \State $S \gets \textsc{RandomSolution}()$
            \State $S \gets \textsc{FastLocalSearch}(S)$
            \If{$f(S) \notin \{f(P) \mid P \in Pop\}$}
                \State $Pop \gets Pop \cup \{S\}$
            \EndIf
        \EndWhile
        \State $S_{best} \gets \textsc{GetBest}(Pop)$
        \State $Stagnation \gets 0$

        \State \textbf{Main Loop:}
        \While{$Elapsed(t) < T_{max}$}
            \State $P_1 \gets \textsc{BinaryTournament}(Pop)$
            \State $P_2 \gets \textsc{Random}(Pop \setminus \{P_1\})$

            \State $C \gets \textsc{EdgeRecombinationCrossover}(P_1, P_2)$

            \If{$Rand(0,1) < 0.05$}
                \State $C \gets \textsc{DoubleBridgeMutation}(C)$
            \EndIf

            \State $C \gets \textsc{FastLocalSearch}(C)$

            \State \Comment{Steady-State Replacement with Diversity Check}
            \If{$f(C) \notin \{f(P) \mid P \in Pop\}$}
                \If{$f(C) < f(\textsc{Worst}(Pop))$}
                    \State Replace $\textsc{Worst}(Pop)$ with $C$
                    \If{$f(C) < f(S_{best})$}
                        \State $S_{best} \gets C$
                        \State $Stagnation \gets 0$
                    \EndIf
                \EndIf
            \Else
                \State $Stagnation \gets Stagnation + 1$
            \EndIf

            \State \Comment{Cataclysm Restart}
            \If{$Stagnation > 300$}
                \State $Pop \gets \{S_{best}\}$
                \While{$|Pop| < N$}
                    \State $S \gets \textsc{RandomSolution}()$
                    \State $S \gets \textsc{FastLocalSearch}(S)$
                    \State $Pop \gets Pop \cup \{S\}$
                \EndWhile
                \State $Stagnation \gets 0$
            \EndIf
        \EndWhile
        \State \Return $S_{best}$
    \end{algorithmic}
\end{algorithm}

\section{Results}

\begin{table}[!htbp]
\centering
\caption{Objective function scores}
\begin{tabular}{lcc}
\hline
 Method & TSPA & TSPB \\
\hline
ils & 69535.9 (69143 - 69964) & 43771 (43463 - 44150) \\
hea & 69247.1 (\textbf{69095} - 69567) & 43638.6 (\textbf{43446} - 44101) \\
hea\_ours & \textbf{69098.6} (\textbf{69095} - \textbf{69137}) & \textbf{43470.7} (\textbf{43446} - \textbf{43536}) \\
\hline
\end{tabular}
\end{table}

\begin{table}[!htbp]
\centering
\caption{Running times in milliseconds}
\begin{tabular}{lcc}
\hline
 Method & TSPA & TSPB \\
\hline
ls\_steepest\_edges\_random & 15.9948 (12.3323 - 27.2404) & 16.31 (12.5805 - 28.0171) \\
ls\_improving\_moves\_list\_random & 11.3866 (7.18689 - 18.2596) & 11.1493 (7.52278 - 16.6551) \\
ls\_improving\_moves\_list\_full\_neighbors\_random & 5.62511 (3.60766 - 9.32537) & 5.36372 (3.6206 - 7.86836) \\
\hline
\end{tabular}
\end{table}

\subsection{TSPA.csv}

\begin{table}[H]
\centering
\caption{Best Achieved Solution for TSPA.csv}
\label{tab:tspa_best_sequence}
\begin{tabular}{|p{0.25\linewidth}|p{0.75\linewidth}|}
\hline
\textbf{Method} & \textbf{Best Sequence} \\
\hline
msls & 121, 53, 180, 154, 135, 70, 127, 123, 112, 4, 190, 10, 177, 54, 48, 34, 160, 184, 35, 131, 149, 65, 116, 43, 42, 181, 146, 22, 159, 193, 41, 139, 115, 46, 68, 93, 117, 143, 0, 153, 183, 89, 186, 23, 137, 176, 51, 118, 59, 162, 151, 133, 80, 79, 63, 94, 124, 167, 148, 9, 62, 102, 144, 14, 49, 3, 178, 106, 52, 55, 185, 40, 119, 165, 27, 90, 81, 196, 145, 78, 31, 113, 175, 171, 16, 25, 44, 120, 92, 57, 129, 2, 75, 86, 101, 1, 152, 97, 26, 100, 121 \\
\hline
ils & 183, 89, 186, 23, 137, 176, 80, 79, 63, 94, 124, 148, 9, 62, 102, 144, 14, 49, 178, 106, 52, 55, 57, 129, 92, 179, 185, 40, 119, 165, 90, 81, 196, 145, 78, 31, 56, 113, 175, 171, 16, 25, 44, 120, 2, 152, 97, 1, 101, 75, 86, 26, 100, 53, 180, 154, 135, 70, 127, 123, 162, 133, 151, 51, 118, 59, 65, 116, 43, 184, 84, 112, 4, 190, 10, 177, 54, 48, 160, 34, 181, 42, 5, 115, 46, 68, 139, 41, 193, 159, 146, 22, 18, 69, 108, 140, 93, 117, 0, 143, 183 \\
\hline
lns\_ls & 53, 180, 154, 135, 70, 127, 123, 162, 133, 151, 51, 118, 59, 65, 116, 43, 42, 184, 35, 84, 112, 4, 190, 10, 177, 54, 48, 160, 34, 181, 146, 22, 18, 108, 69, 159, 193, 41, 139, 115, 46, 68, 140, 93, 117, 0, 143, 183, 89, 186, 23, 137, 176, 80, 79, 63, 94, 124, 148, 9, 62, 102, 144, 14, 49, 178, 106, 52, 55, 57, 129, 92, 78, 145, 179, 185, 40, 165, 90, 81, 196, 157, 31, 56, 113, 175, 171, 16, 25, 44, 120, 2, 152, 97, 1, 101, 75, 86, 26, 100, 53 \\
\hline
hea\_op1 & 31, 56, 113, 175, 171, 16, 25, 44, 120, 2, 152, 97, 1, 101, 75, 86, 26, 100, 53, 180, 154, 135, 70, 127, 123, 162, 133, 151, 51, 118, 59, 65, 116, 43, 42, 184, 35, 84, 112, 4, 190, 10, 177, 54, 48, 160, 34, 181, 146, 22, 159, 193, 41, 139, 115, 46, 68, 69, 18, 108, 140, 93, 117, 0, 143, 183, 89, 186, 23, 137, 176, 80, 79, 63, 94, 124, 148, 9, 62, 102, 144, 14, 49, 178, 106, 52, 55, 57, 129, 92, 179, 185, 40, 119, 165, 90, 81, 196, 145, 78, 31 \\
\hline
hea\_op2\_nols & 127, 123, 162, 151, 133, 79, 80, 176, 51, 118, 59, 65, 116, 43, 42, 181, 160, 184, 35, 84, 112, 4, 190, 10, 177, 54, 48, 34, 146, 22, 18, 108, 69, 159, 193, 41, 139, 115, 46, 68, 140, 93, 117, 0, 143, 183, 89, 23, 137, 148, 9, 62, 102, 144, 14, 49, 178, 106, 52, 55, 57, 185, 40, 165, 39, 90, 81, 196, 179, 145, 78, 31, 56, 113, 175, 171, 16, 25, 44, 120, 82, 92, 129, 2, 75, 101, 86, 100, 26, 97, 1, 152, 124, 94, 63, 53, 180, 154, 135, 70, 127 \\
\hline
hea\_op2\_ls & 34, 181, 146, 22, 18, 108, 69, 159, 193, 41, 139, 115, 46, 68, 140, 93, 117, 0, 143, 183, 89, 23, 137, 176, 80, 79, 63, 94, 124, 148, 9, 62, 144, 14, 49, 178, 106, 52, 55, 185, 40, 165, 90, 81, 196, 157, 31, 56, 113, 175, 171, 16, 25, 44, 120, 78, 145, 179, 57, 92, 129, 2, 152, 97, 1, 101, 75, 86, 26, 100, 121, 53, 180, 154, 135, 70, 127, 123, 162, 133, 151, 51, 118, 59, 149, 65, 116, 43, 42, 184, 35, 84, 112, 4, 190, 10, 177, 54, 48, 160, 34 \\
\hline
\end{tabular}
\end{table}

\begin{figure}[!htbp]
\centering
\includegraphics[width=0.97\linewidth]{TSPA_solution_msls.png}
\caption{\label{fig:tspa_msls}msls}
\end{figure}

\begin{figure}[!htbp]
\centering
\includegraphics[width=0.97\linewidth]{TSPA_solution_ils.png}
\caption{\label{fig:tspa_ils}ils}
\end{figure}

\begin{figure}[!htbp]
\centering
\includegraphics[width=0.97\linewidth]{TSPA_solution_lns_ls.png}
\caption{\label{fig:tspa_lns_ls}lns\_ls}
\end{figure}

\begin{figure}[!htbp]
\centering
\includegraphics[width=0.97\linewidth]{TSPA_solution_hea_op1.png}
\caption{\label{fig:tspa_hea_op1}hea\_op1}
\end{figure}

\begin{figure}[!htbp]
\centering
\includegraphics[width=0.97\linewidth]{TSPA_solution_hea_op2_nols.png}
\caption{\label{fig:tspa_hea_op2_nols}hea\_op2\_nols}
\end{figure}

\begin{figure}[!htbp]
\centering
\includegraphics[width=0.97\linewidth]{TSPA_solution_hea_op2_ls.png}
\caption{\label{fig:tspa_hea_op2_ls}hea\_op2\_ls}
\end{figure}


\subsection{TSPB.csv}

\begin{table}[H] % Use [H] from float package for 'here' if available, otherwise use [!htbp]
\centering
\caption{Best Achieved Solution for TSPB.csv}
\label{tab:tsbp_best_sequence}
\begin{tabular}{|p{0.3\linewidth}|p{0.7\linewidth}|} % Two columns: 25% for Method, 75% for Sequence
\hline
\textbf{Method} & \textbf{Best Sequence} \\
\hline
ls\_steepest\_edges\_random & 47, 94, 66, 179, 185, 95, 86, 166, 194, 176, 180, 113, 103, 127, 89, 163, 153, 81, 77, 141, 61, 36, 177, 5, 78, 175, 142, 45, 80, 190, 136, 73, 164, 54, 31, 193, 117, 198, 156, 1, 27, 38, 131, 121, 51, 125, 191, 90, 122, 135, 63, 40, 107, 133, 10, 178, 147, 6, 188, 169, 132, 161, 70, 3, 15, 145, 13, 126, 195, 168, 139, 11, 182, 138, 33, 160, 29, 109, 35, 0, 144, 56, 104, 8, 21, 82, 111, 143, 106, 124, 62, 83, 18, 55, 34, 152, 183, 140, 20, 148, 47 \\
\hline
ls\_improving\_moves\_list\_random & 98, 51, 121, 90, 147, 115, 133, 122, 135, 32, 63, 38, 27, 1, 156, 198, 30, 117, 193, 31, 54, 73, 136, 190, 80, 45, 142, 175, 78, 5, 177, 25, 138, 104, 56, 8, 82, 21, 61, 36, 91, 141, 97, 77, 81, 153, 187, 163, 89, 127, 103, 113, 176, 194, 166, 86, 185, 179, 94, 47, 148, 60, 20, 28, 140, 183, 95, 106, 124, 62, 18, 55, 34, 170, 152, 184, 155, 3, 70, 15, 145, 13, 132, 169, 188, 6, 195, 168, 29, 0, 109, 35, 111, 144, 160, 33, 11, 139, 134, 118, 98 \\
\hline
\end{tabular}
\end{table}

\begin{figure}[!htbp]
\centering
\includegraphics[width=0.97\linewidth]{TSPB_solution_ls_steepest_edges_random.png}
\caption{\label{fig:frog}ls\_steepest\_edges\_random}
\end{figure}

\begin{figure}[!htbp]
\centering
\includegraphics[width=0.97\linewidth]{TSPB_solution_ls_improving_moves_list_random.png}
\caption{\label{fig:frog}ls\_improving\_moves\_list\_random}
\end{figure}

\newpage

\section{Conclusions}

The local search algorithms, particularly those employing a greedy initial solution, demonstrate a significant improvement over simple constructive heuristics. The quality of the starting solution is crucial, as evidenced by the consistently better performance of methods with a greedy start compared to a random start.

Among the neighborhood structures, the edge-based neighborhood (2-opt) proves to be more effective and efficient than the node-based neighborhood. This is reflected in both the solution quality and the computation times.

The `ls\_steepest\_edges\_greedy` method stands out as the most effective algorithm, achieving the best scores for both TSPA and TSPB instances while also being one of the fastest. This suggests that a steepest search strategy combined with an edge-based neighborhood and a good initial solution is a powerful combination for this problem.

In general, the results highlight the importance of a well-chosen combination of initial solution generation, neighborhood structure, and search strategy for solving the TSP effectively.

Link to the source code:

\url{https://github.com/markopolo139/Evolutionary-computation-put/tree/main/ass3}

The best solutions were checked with the provided solution checker.

\end{document}
